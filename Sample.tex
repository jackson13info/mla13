\documentclass{article}
\usepackage{mla13}
\title{Sample Sources MLA13 Document}
\firstname{John}
\lastname{Smith}
\professor{Dr. Professor}
\class{FunClass}
\sources{SampleBibDocument.bib}
\begin{document}
\makeheader

In a recent report by the United Nations, they found that more than 884 million people do not have access to safe drinking water \cite[e.g.][15-23]{unWater}. This number equates to more than 1 out of every 8 people, not having access to something that is so vital to human life. Knowing this fact, most must ask themselves, why the same water that we drink is used to clean our toilets and wash our lawns. The water that hundreds of millions of people would love to have, is something that we just flush down the toilet. This paper intends to examine the benefits of grey water systems, and how their use leads to increased water conservation efforts, creating more benefits then costs.

Grey water systems are a technique that aids in water conservation efforts by reusing water that doesn't need to be fully cleaned. For example, many grey water systems use the water that comes from the shower drain to water the lawn or fill the toilet. Even though this water isn't going to a water treatment plant, doesn't mean that it is not clean. Grey water systems are equipped with a filter that removes most soaps and solid objects that make their way through the drainage system \cite{planetArk}. With a private market for greywater systems developing, there are a variety of commercial systems that filter water to ``remove hair, lint, and debris, and remove pollutants, bacteria, salts'' and many more materials \cite{pacificInst2010}

\makeworkscited
\end{document}
